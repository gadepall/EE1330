\documentclass[12pt,onecolumn]{IEEEtran}
\usepackage[latin1]{inputenc}
%\usepackage{graphicx}
\usepackage{amsmath}
\usepackage{amscd}
\usepackage{amsfonts}
\usepackage{amssymb}
\usepackage{amsthm}
\usepackage{mathrsfs}
\usepackage{stfloats}
\usepackage{listings}
\usepackage{iithquiz}
%\usepackage{tfrupee}
%\usepackage{multicol}
\usepackage{array}
\DeclareMathOperator*{\Res}{Res}
%\DeclareMathOperator*{\pe}{\Pr}
%\usepackage{amsmath, verbatim,url,graphicx,pxfonts,setspace,fancyhdr}

\newtheorem{theorem}{Theorem}[section]
\newtheorem{proposition}{Proposition}[section]
\newtheorem{lemma}{Lemma}[section]
%\newtheorem{example}{Example}[section]
\newtheorem{example}{Example}
\theoremstyle{definition}
\newtheorem{problem}{Problem}
\newtheorem{cprob}{Challenging Problem}
\newtheorem{computer}{Computer Exercise}
\newtheorem{definition}{Definition}[section]
\newtheorem{algorithm}{Algorithm}[section]
\theoremstyle{remark}
\newtheorem{rem}{Remark}

\providecommand{\abs}[1]{\lvert#1\rvert}
\providecommand{\res}[1]{\Res\displaylimits_{#1}} 
\providecommand{\norm}[1]{\lVert#1\rVert}
\providecommand{\mtx}[1]{\mathbf{#1}}
\providecommand{\mean}[1]{E\left[ #1 \right]}
\providecommand{\fourier}{\overset{\mathcal{F}}{ \rightleftharpoons}}
%\providecommand{\hilbert}{\overset{\mathcal{H}}{ \rightleftharpoons}}
\providecommand{\system}{\overset{\mathcal{H}}{ \longleftrightarrow}}
%\newcommand{\solution}[2]{\textbf{Solution:}{#1}}
\newcommand{\solution}{\noindent \textbf{Solution: }}
\providecommand{\pr}[1]{\ensuremath{\Pr\left(#1\right)}}
\providecommand{\qfunc}[1]{\ensuremath{Q\left(#1\right)}}
\providecommand{\sbrak}[1]{\ensuremath{{}\left[#1\right]}}
\providecommand{\lsbrak}[1]{\ensuremath{{}\left[#1\right.}}
\providecommand{\rsbrak}[1]{\ensuremath{{}\left.#1\right]}}
\providecommand{\brak}[1]{\ensuremath{\left(#1\right)}}
\providecommand{\lbrak}[1]{\ensuremath{\left(#1\right.}}
\providecommand{\rbrak}[1]{\ensuremath{\left.#1\right)}}
\providecommand{\cbrak}[1]{\ensuremath{\left\{#1\right\}}}
\providecommand{\lcbrak}[1]{\ensuremath{\left\{#1\right.}}
\providecommand{\rcbrak}[1]{\ensuremath{\left.#1\right\}}}
%\providecommand{\rbrak}[1]{\ensuremath{\left. #1\right \}}}
%\providecommand{\curly}[1]{\ensuremath{\left\{#1\right\}}}

\bibliographystyle{IEEEtran}

\begin{document}

\title{
\logo{Vidya Jyothi Institute of Technology}{\begin{center}\includegraphics[scale=.24]{tlc}\end{center}}{}{HAMDSP}
}
\maketitle
\vspace{-1cm}
\textit{Problem 1:}  To separate two sinusoidal signals of frequencies 1 kHz and 2 kHz.
\begin{enumerate}
\item Sketch the spectrum of 
\\
$s_1(t)=\cos\brak{2\pi f_1t}$ and
\\ 
$s_2(t)=\cos\brak{2\pi f_2t}$ 
\\
where, $f_1$ = 1 kHz  and $f_2$ = 2 kHz
\item Let $y(t)=s_1(t)+s_2(t)$.
 Obtain the filter $h(t)$ that can be used to get $s_1(t)$ from $y(t)$.
\end{enumerate}
\vspace{1cm}
\textit{Problem 2:} Let $H(z)=G\frac{1-z^{-2}}{1-2rcos(\omega_0)z^{-1}+r^2z^{-2}}$
where $r=0.95, \omega_0=\pi/3$ and $max \,\{ \abs{H(e^{j\omega})}\}=1$
\begin{enumerate}
\item Find G.
\item Write the input output relation for this discrete system.
\item Plot $\abs{H(e^{\j\omega})}$.
\item Find the impulse response for the given system.
\end{enumerate}
\vspace{1cm}
\textit{Problem 3:} Consider $x_1(n)$=\{1,1,1,1\}
\begin{enumerate}
\item Find $x_1(n)*x_1(-n)$.
\item Let $x_2(n)$=\{1,1,-1,-1\},
find $x_1(n)*x_2(-n)$.
\item Let $x_3(n)$=\{1,1,1,-1\},
find $x_1(n)*x_3(-n)$.
\end{enumerate} 
%
%\begin{center}
%   \scalebox{1}{%
%   \normalsize
%   \parbox{6.57292in}{%
%\includegraphics[scale=0.7]{quiz1_prob1.eps} \\
%   % translate x=960 y=544 scale 0.38
%   \putbox{3.0in}{2.5in}{1.20}{$V$}%
%    } % close 'parbox'
%   } % close 'scalebox'
%   \vspace{-\baselineskip} % this is not necessary, but looks better
%\end{center}
%


\end{document}
