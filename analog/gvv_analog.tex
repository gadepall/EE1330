\documentclass[journal,12pt,twocolumn]{IEEEtran}
%
\usepackage{setspace}
\usepackage{gensymb}
%\usepackage{graphicx}
\usepackage{xcolor}
\usepackage{caption}
%\usepackage{subcaption}
\usepackage{tabularx}
\usepackage{booktabs}
\usepackage{url}
%\doublespacing
\singlespacing

%\usepackage{graphicx}
%\usepackage{amssymb}
%\usepackage{relsize}
\usepackage[cmex10]{amsmath}
\usepackage{mathtools}

%\usepackage{amsthm}
%\interdisplaylinepenalty=2500
%\savesymbol{iint}
%\usepackage{txfonts}
%\restoresymbol{TXF}{iint}
%\usepackage{wasysym}
%\usepackage{tikz}
%\usetikzlibrary{matrix,shapes,arrows,positioning,chains}
%\usetikzlibrary{intersections}
\usepackage{textcomp}
\usepackage{amsthm}
\usepackage{mathrsfs}
\usepackage{txfonts}
\usepackage{stfloats}
\usepackage{cite}
\usepackage{cases}
\usepackage{subfig}
%\usepackage{xtab}
\usepackage{longtable}
\usepackage{multirow}
%\usepackage{algorithm}
%\usepackage{algpseudocode}
\usepackage{enumitem}
\usepackage{mathtools}
\usepackage{iithtlc}
%\usepackage[framemethod=tikz]{mdframed}
\usepackage{listings}
\usepackage{listings}
    \usepackage[latin1]{inputenc}                                 %%
    \usepackage{color}                                            %%
    \usepackage{array}                                            %%
    \usepackage{longtable}                                        %%
    \usepackage{calc}                                             %%
    \usepackage{multirow}                                         %%
    \usepackage{hhline}                                           %%
    \usepackage{ifthen}                                           %%
  %optionally (for landscape tables embedded in another document): %%
    \usepackage{lscape}     



%\usepackage{stmaryrd}


%\usepackage{wasysym}
%\newcounter{MYtempeqncnt}
\DeclareMathOperator*{\Res}{Res}
%\renewcommand{\baselinestretch}{2}
\renewcommand\thesection{\arabic{section}}
\renewcommand\thesubsection{\thesection.\arabic{subsection}}
\renewcommand\thesubsubsection{\thesubsection.\arabic{subsubsection}}

\renewcommand\thesectiondis{\arabic{section}}
\renewcommand\thesubsectiondis{\thesectiondis.\arabic{subsection}}
\renewcommand\thesubsubsectiondis{\thesubsectiondis.\arabic{subsubsection}}

% correct bad hyphenation here
\hyphenation{op-tical net-works semi-conduc-tor}

\lstset{
language=Python,
frame=single, 
breaklines=true
}

%\lstset{
	%%basicstyle=\small\ttfamily\bfseries,
	%%numberstyle=\small\ttfamily,
	%language=Octave,
	%backgroundcolor=\color{white},
	%%frame=single,
	%%keywordstyle=\bfseries,
	%%breaklines=true,
	%%showstringspaces=false,
	%%xleftmargin=-10mm,
	%%aboveskip=-1mm,
	%%belowskip=0mm
%}

%\surroundwithmdframed[width=\columnwidth]{lstlisting}
\def\inputGnumericTable{}                                 %%
\lstset{
language=Python,
frame=single, 
breaklines=true
}
 
%\usepackage{hyperref}
\begin{document}
%

\theoremstyle{definition}
\newtheorem{theorem}{Theorem}[section]
\newtheorem{problem}{Problem}
\newtheorem{proposition}{Proposition}[section]
\newtheorem{lemma}{Lemma}[section]
\newtheorem{corollary}[theorem]{Corollary}
\newtheorem{example}{Example}[section]
\newtheorem{definition}{Definition}[section]
%\newtheorem{algorithm}{Algorithm}[section]
%\newtheorem{cor}{Corollary}
\newcommand{\BEQA}{\begin{eqnarray}}
\newcommand{\EEQA}{\end{eqnarray}}
\newcommand{\define}{\stackrel{\triangle}{=}}

\bibliographystyle{IEEEtran}
%\bibliographystyle{ieeetr}

\providecommand{\nCr}[2]{\,^{#1}C_{#2}} % nCr
\providecommand{\nPr}[2]{\,^{#1}P_{#2}} % nPr
\providecommand{\mbf}{\mathbf}
\providecommand{\rect}{\text{rect}}
\providecommand{\sinc}{\text{sinc}}
\providecommand{\pr}[1]{\ensuremath{\Pr\left(#1\right)}}
\providecommand{\qfunc}[1]{\ensuremath{Q\left(#1\right)}}
\providecommand{\sbrak}[1]{\ensuremath{{}\left[#1\right]}}
\providecommand{\lsbrak}[1]{\ensuremath{{}\left[#1\right.}}
\providecommand{\rsbrak}[1]{\ensuremath{{}\left.#1\right]}}
\providecommand{\brak}[1]{\ensuremath{\left(#1\right)}}
\providecommand{\lbrak}[1]{\ensuremath{\left(#1\right.}}
\providecommand{\rbrak}[1]{\ensuremath{\left.#1\right)}}
\providecommand{\cbrak}[1]{\ensuremath{\left\{#1\right\}}}
\providecommand{\lcbrak}[1]{\ensuremath{\left\{#1\right.}}
\providecommand{\rcbrak}[1]{\ensuremath{\left.#1\right\}}}
\theoremstyle{remark}
\newtheorem{rem}{Remark}
\newcommand{\sgn}{\mathop{\mathrm{sgn}}}
\providecommand{\abs}[1]{\left\vert#1\right\vert}
\providecommand{\res}[1]{\Res\displaylimits_{#1}} 
\providecommand{\norm}[1]{\lVert#1\rVert}
\providecommand{\mtx}[1]{\mathbf{#1}}
\providecommand{\mean}[1]{E\left[ #1 \right]}
\providecommand{\fourier}{\overset{\mathcal{F}}{ \rightleftharpoons}}
\providecommand{\ztrans}{\overset{\mathcal{Z}}{ \rightleftharpoons}}
%\providecommand{\hilbert}{\overset{\mathcal{H}}{ \rightleftharpoons}}
\providecommand{\system}{\overset{\mathcal{H}}{ \longleftrightarrow}}
	%\newcommand{\solution}[2]{\textbf{Solution:}{#1}}
\newcommand{\solution}{\noindent \textbf{Solution: }}
\providecommand{\dec}[2]{\ensuremath{\overset{#1}{\underset{#2}{\gtrless}}}}
%\numberwithin{equation}{subsection}
\numberwithin{equation}{problem}
%\numberwithin{problem}{subsection}
%\numberwithin{definition}{subsection}
\makeatletter
\@addtoreset{figure}{problem}
\makeatother

\let\StandardTheFigure\thefigure
%\renewcommand{\thefigure}{\theproblem.\arabic{figure}}
\renewcommand{\thefigure}{\theproblem}


%\numberwithin{figure}{subsection}

%\numberwithin{equation}{subsection}
%\numberwithin{equation}{section}
%%\numberwithin{equation}{problem}
%%\numberwithin{problem}{subsection}
%\numberwithin{problem}{section}
%%\numberwithin{definition}{subsection}
%\makeatletter
%\@addtoreset{figure}{problem}
%\makeatother
%\makeatletter
%\@addtoreset{table}{problem}
%\makeatother

%\let\StandardTheFigure\thefigure
%\let\StandardTheTable\thetable
%%\renewcommand{\thefigure}{\theproblem.\arabic{figure}}
%\renewcommand{\thefigure}{\theproblem}
%\renewcommand{\thetable}{\theproblem}
%%\numberwithin{figure}{section}

%%\numberwithin{figure}{subsection}



\def\putbox#1#2#3{\makebox[0in][l]{\makebox[#1][l]{}\raisebox{\baselineskip}[0in][0in]{\raisebox{#2}[0in][0in]{#3}}}}
     \def\rightbox#1{\makebox[0in][r]{#1}}
     \def\centbox#1{\makebox[0in]{#1}}
     \def\topbox#1{\raisebox{-\baselineskip}[0in][0in]{#1}}
     \def\midbox#1{\raisebox{-0.5\baselineskip}[0in][0in]{#1}}

\vspace{3cm}

\title{
	\logo{
Sampling and Reconstruction
	}
}
%\title{
%	\logo{Matrix Analysis through Octave}{\begin{center}\includegraphics[scale=.24]{tlc}\end{center}}{}{HAMDSP}
%}


% paper title
% can use linebreaks \\ within to get better formatting as desired
%\title{Matrix Analysis through Octave}
%
%
% author names and IEEE memberships
% note positions of commas and nonbreaking spaces ( ~ ) LaTeX will not break
% a structure at a ~ so this keeps an author's name from being broken across
% two lines.
% use \thanks{} to gain access to the first footnote area
% a separate \thanks must be used for each paragraph as LaTeX2e's \thanks
% was not built to handle multiple paragraphs
%

\author{ G V V Sharma$^{*}$ %<-this  stops a space
\thanks{*The author is with the Department
of Electrical Engineering, Indian Institute of Technology, Hyderabad
502285 India e-mail:  gadepall@iith.ac.in.}% <-this % stops a space
%\thanks{J. Doe and J. Doe are with Anonymous University.}% <-this % stops a space
%\thanks{Manuscript received April 19, 2005; revised January 11, 2007.}}
}
% note the % following the last \IEEEmembership and also \thanks - 
% these prevent an unwanted space from occurring between the last author name
% and the end of the author line. i.e., if you had this:
% 
% \author{....lastname \thanks{...} \thanks{...} }
%                     ^------------^------------^----Do not want these spaces!
%
% a space would be appended to the last name and could cause every name on that
% line to be shifted left slightly. This is one of those "LaTeX things". For
% instance, "\textbf{A} \textbf{B}" will typeset as "A B" not "AB". To get
% "AB" then you have to do: "\textbf{A}\textbf{B}"
% \thanks is no different in this regard, so shield the last } of each \thanks
% that ends a line with a % and do not let a space in before the next \thanks.
% Spaces after \IEEEmembership other than the last one are OK (and needed) as
% you are supposed to have spaces between the names. For what it is worth,
% this is a minor point as most people would not even notice if the said evil
% space somehow managed to creep in.



% The paper headers
%\markboth{Journal of \LaTeX\ Class Files,~Vol.~6, No.~1, January~2007}%
%{Shell \MakeLowercase{\textit{et al.}}: Bare Demo of IEEEtran.cls for Journals}
% The only time the second header will appear is for the odd numbered pages
% after the title page when using the twoside option.
% 
% *** Note that you probably will NOT want to include the author's ***
% *** name in the headers of peer review papers.                   ***
% You can use \ifCLASSOPTIONpeerreview for conditional compilation here if
% you desire.




% If you want to put a publisher's ID mark on the page you can do it like
% this:
%\IEEEpubid{0000--0000/00\$00.00~\copyright~2007 IEEE}
% Remember, if you use this you must call \IEEEpubidadjcol in the second
% column for its text to clear the IEEEpubid mark.



% make the title area
\maketitle

%\newpage

\tableofcontents

\bigskip

%\begin{abstract}
%%\boldmath
%In this letter, an algorithm for evaluating the exact analytical bit error rate  (BER)  for the piecewise linear (PL) combiner for  multiple relays is presented. Previous results were available only for upto three relays. The algorithm is unique in the sense that  the actual mathematical expressions, that are prohibitively large, need not be explicitly obtained. The diversity gain due to multiple relays is shown through plots of the analytical BER, well supported by simulations. 
%
%\end{abstract}
\begin{abstract}
%\boldmath
This manual provides theoretical insights into analog to digital (ADC) and digital to analog (DAC) conversion.
\end{abstract}

\section{Fourier Transform}
\begin{problem}
Let $x(t)$ be a continuous signal with $x(n) = x(nT_s)$ and
\begin{equation}
\hat{x}(t) = \sum_{n=-\infty}^{\infty}x(n)\delta\brak{t-nT_s},
\end{equation}
where
\begin{align}
\int_{-\infty}^{\infty}\delta(t) \, dt &= 1
\\
\delta(t) &=0, t \ne 0
\end{align}
%
Show that
\begin{equation}
\hat{x}(t) = \sum_{n=-\infty}^{\infty}x(t)\delta\brak{t-nT_s},
\end{equation}
%\begin{equation}
%x(n) = x(nT_s),
%\end{equation}
%where $T_s$ is the sampling time.  Obtain $x(t)$ from $x(n)$.
\end{problem}
\begin{problem}
The {\em Fourier transform} of a signal $g(t)$ is defined as
\begin{equation}
G(f) = \int_{-\infty}^{\infty}g(t)e^{-\j 2\pi f t}\, dt
\end{equation}
%
Find the Fourier transform of $\delta(t)$ and show that 
\begin{equation}
\delta\brak{t-nT_s} \fourier e^{-\j2\pi nf T_s}
\end{equation}
\end{problem}
\begin{problem}
Find the Fourier transform of $\hat{x}(t)$.
\end{problem}
\begin{problem}
If
\begin{equation}
x(t) \fourier X(f),
\end{equation}
the {\em inverse} Fourier transform is given by
\begin{equation}
\label{eq:inv_ft}
x(t) = \int_{-\infty}^{\infty}X(f)e^{\j 2\pi ft}\, df
\end{equation}

Show that 
\begin{equation}
\label{eq:ft}
x(n) = \int_{-\infty}^{\infty}X(f)e^{\j 2\pi n fT_s}\, df
\end{equation}
\end{problem}
\section{The Fourier Series}
\begin{problem}
Let
\begin{equation}
\hat{X}(f) = \sum_{n=-\infty}^{\infty}x(n)e^{-\j 2\pi nfT_s}
\end{equation}
%
Show that 
\begin{equation}
\hat{X}(f) = \hat{X}\brak{f+f_s}, \quad f_s = \frac{1}{T_s}
\end{equation}
\end{problem}
%
\begin{problem}
Show that
\begin{equation}
\label{eq:inv_fs}
x(n) = \frac{1}{f_s}\int_{-\frac{f_s}{2}}^{\frac{f_s}{2}}\hat{X}(f)e^{\j 2\pi nfT_s}\, df
\end{equation}
\end{problem}
\begin{problem}
Use your intuition along with  \eqref{eq:ft} and \eqref{eq:inv_fs} to obtain
\begin{equation}
\hat{X} = \frac{X(f)}{T_s}, \quad -\frac{f_s}{2} < f < \frac{f_s}{2}
\end{equation}
and
\begin{equation}
\label{eq:dtft}
\hat{X}(f) = \frac{1}{T_s}\sum_{n=-\infty}^{\infty}X(f-nf_s)
\end{equation}
\end{problem}
\section{Nyquist Criterion}
\begin{problem}
Let 
\begin{equation}
X(f) = 
\begin{cases}
1 - \frac{\abs{f}}{B} & \abs{f} < B
\\
0 & \text{otherwise}
\end{cases},
B = \alpha f_s
\end{equation}
Using \eqref{eq:dtft},  sketch $\hat{X}(f)$ for $\alpha < 2$ after fixing a particular value of $B$. 
\end{problem}
\begin{problem}
Repeat the above exercise for $\alpha < 2$.  Comment.  
\end{problem}
\begin{problem}
Let 
\begin{equation}
H(f) = 
\begin{cases}
T_s & \abs{f} < f_s
\\
0 & \text{otherwise}
\end{cases}
\end{equation}
Find the constraint on $\alpha$ that yields
\begin{equation}
X(f) = \hat{X}(f) H(f)
\end{equation}
This is known as Nyquist's criterion.
\end{problem}
\section{Shannon's Interpolation Formula}
\begin{problem}
{\em Convolution} of $\hat{x}(t)$ and $h(t)$ is defined as
\begin{equation}
\hat{x}(t)*h(t) \define \int_{-\infty}^{\infty}\hat{x}(\tau)h(t -\tau) \, d\tau
\end{equation}
Show that
\begin{equation}
\hat{x}(t)*h(t) \fourier \hat{X}(f)H(f)
\end{equation}
\end{problem}
\begin{problem}
Show that
\begin{equation}
h(t)*\delta\brak{t-nT_s} = h\brak{t-nT_s}
\end{equation}
\end{problem}
\begin{problem}
Find $h(t)$ from $H(f)$ using \eqref{eq:inv_ft} and sketch it.
\end{problem}
\begin{problem}
Show that
\begin{equation}
x(t) = \sum_{n = -\infty}^{\infty} x(n) \sinc \brak{t - nT_s}
\end{equation}
where
\begin{equation}
\sinc(t) = \frac{\sin \pi t}{\pi t}
\end{equation}
\end{problem}
%\begin{align}
%x(t) &= \rect\brak{t} = 
%\begin{cases}
%1 & t \in \brak{-\frac{1}{2},\frac{1}{2}}
%\\
%0 & \text{otherwise}
%\end{cases}
%\end{align}
%Sketch the {\em Fourier transform} of $x(t)$, defined as
%\begin{equation}
%X(f) = \int_{-\infty}^{\infty}x(t)e^{-\j 2\pi f t}\,dt
%\end{equation}
%\end{problem}
\end{document}


